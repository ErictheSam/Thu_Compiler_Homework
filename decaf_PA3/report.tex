\documentclass[UTF8]{ctexart}
\usepackage{listings}
\usepackage[usenames,dvipsnames]{xcolor}
\definecolor{mygreen}{rgb}{0,0.6,0}
\definecolor{mygray}{rgb}{0.5,0.5,0.5}
\definecolor{mymauve}{rgb}{0.58,0,0.82}
\lstset{
 backgroundcolor=\color{lightgray}, 
 basicstyle = \footnotesize,       
 breakatwhitespace = false,        
 breaklines = true,                 
 captionpos = b,                    
 commentstyle = \color{mygreen}\bfseries,
 extendedchars = false,             
 frame =shadowbox, 
 framerule=0.5pt,
 keepspaces=true,
 keywordstyle=\color{blue}\bfseries, % keyword style
 language = C++,                     % the language of code
 otherkeywords={string}, 
 numbers=left, 
 numbersep=5pt,
 numberstyle=\tiny\color{mygray},
 rulecolor=\color{black},         
 showspaces=false,  
 showstringspaces=false, 
 showtabs=false,    
 stepnumber=1,         
 stringstyle=\color{mymauve},        % string literal style
 tabsize=2,          
 title=\lstname                      
}
\begin{document}
\begin{center}
\begin{LARGE}
PA3实验报告\\
\end{LARGE}
计76 沈诣博 2017011427
\end{center}
本阶段的工作分为如下几部分:\\
主要部分:修改了提供的translate部分和error部分代码,增添了对于新特性的支持。\\
其他部分:在PA2基础上针对两次PA特性的不一致(foreach语句)部分进行了修改。\\
首先说说主要部分。我修改了translate部分的代码,实现了浅复制,卫士语句,变量类型推导,数组特性和除0报错的实现。\\\\
\textbf{1.浅复制}\\\\
浅复制需要如下几步:
首先,注意到Scope类里有一个包含域的所有成员标识的Map<String, Symbol> symbols,在需要知晓一个类的全部成员时可以调用symbols.variables()进行遍历。\\
然后可以将浅复制的过程分为以下部分:\\
1.算出新建该类对象的空间并进行申请:\\\\
一个类对象实体需要的空间大小是4$\times$类成员的总数,因此在第一次遍历类成员的时候算出该实体的空间,并且调用alloc库函数进行申请。\\
2.找到每一个成员的地址并进行复制:\\\\
由于在Transpass1类里已经对于对于任意一个类的成员的Variable标识赋予了指针偏置量,在浅复制的时候只需遍历他们进行偏置即可找到相应的寄存器。对于每一个寄存器内的内容进行复制。\\
3.将申请空间的头指针地址给存进需要浅复制的对象所处的寄存器内。\\\\
\textbf{2.条件卫士语句}\\\\
对条件卫士语句的每一条子语句进行遍历,对于子语句的遍历过程如下:\\
1.生成一个“结束”标签,如果子语句条件部分为负就跳至标签处\\
2.遍历执行语句,并将标签附在执行语句之后。\\\\
\textbf{3.变量类型推导}\\\\
支持变量类型推导的修改比较简单:由于已经在第二阶段实现语法检测,直接对于相关的变量新增tac即可。\\\\
\textbf{4.支持一维数组}\\\\
1.对于数组初始化常量表达式,我使用了以下的方式生成中间代码:\\
首先,检查新增数组的大小是否小于等于0,是则报错退出,否则申请空间并继续。\\
然后,检查数组元素的类型是否是class,是则调用浅复制函数,否则直接复制。\\
在本段代码的最后,生成一个退出标签;在复制元素之前,生成一个循环标签并从后到前进行数组的逐元素复制。在一轮复制之前进行判断,若需要复制的元素偏置小于0,则跳到退出标签,否则复制完之后跳到循环标签。\\\\
2.对于数组下标动态访问表达式,我使用了以下的方式生成中间代码:\\
对于返回的不同值,设置两个标签,标出两个不同的中间代码分支:\\
首先,检查下标是否合法:获取数组E的大小,并与下标进行比较,若下标小于0或者大于数组E的体积,则跳到第二个分支,否则跳到第一个分支。\\
第一个分支返回数组相应元素,而第二个分支返回表达式的值。\\\\
3.对于数组数组迭代语句,我采用了以下的方式生成中间代码:\\
首先,生成一个loop循环,在循环开始之前查看当前绑定的值下标是否在范围内,是则将迭代值x绑定为数组的相关元素,否则跳到exit标签。\\
然后,若有while条件判断语句,则遍历该语句,如果值为假则跳到exit标签。\\
最后,遍历代码块S。\\\\
\textbf{实现除0错误的识别}\\\\
我对translate类的genDiv()和genMod()方法进行了如下的改变:\\
首先判断被除/模数是否为0,是则报错退出,否则继续。\\
此外,对于error文件夹内的RunTimeError类,我增加了除0报错和数组初始化报错两条,从而支持示例类的具体报错语句。\\\\
对于次要部分进行的修改如下:\\\\
1.将PA2中的frontend,tree以及typecheck复制进来,并且根据复制进来代码实现了sealed特性。\\
2.在tree.java中,我在Expr子类里添加了标识中间码的tac变量,即对于原有的tree.java的新加特性进行merge。\\
3.修改了foreach语句的特性,使之支持条件语句缺省的情况,并且修复了PA2中未能支持break语句的bug。\\
\end{document}