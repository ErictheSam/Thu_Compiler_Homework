\documentclass[UTF8]{ctexart}
\usepackage{listings}
\usepackage[usenames,dvipsnames]{xcolor}
\definecolor{mygreen}{rgb}{0,0.6,0}
\definecolor{mygray}{rgb}{0.5,0.5,0.5}
\definecolor{mymauve}{rgb}{0.58,0,0.82}
\lstset{
 backgroundcolor=\color{lightgray}, 
 basicstyle = \footnotesize,       
 breakatwhitespace = false,        
 breaklines = true,                 
 captionpos = b,                    
 commentstyle = \color{mygreen}\bfseries,
 extendedchars = false,             
 frame =shadowbox, 
 framerule=0.5pt,
 keepspaces=true,
 keywordstyle=\color{blue}\bfseries, % keyword style
 language = C++,                     % the language of code
 otherkeywords={string}, 
 numbers=left, 
 numbersep=5pt,
 numberstyle=\tiny\color{mygray},
 rulecolor=\color{black},         
 showspaces=false,  
 showstringspaces=false, 
 showtabs=false,    
 stepnumber=1,         
 stringstyle=\color{mymauve},        % string literal style
 tabsize=2,          
 title=\lstname                      
}
\begin{document}
\begin{center}
PA4实验报告\\
计76 沈诣博 2017011427
\end{center}
\textbf{本阶段的工作}\\
本人在原有框架上增加了求解DU链的方法和接口,改动文件如下:\\
1.dataflow/BasicBlock.java:\\
增加了函数computeDUDefAndLiveUse(Map<Temp,Set<BasicBlock>>)初始化修改定义之后的LiveUse集合,并且求出基本块内的DU链。\\
增加了函数computeDEFED(Map<Temp,Set<BasicBlock>>),使用画家算法求出修改定义之后的Def集合。\\
增加了函数analyzeDUChain()添加块和块之间的DU链。\\\\
2.dataflow/FlowGraph.java:\\
增加了函数analyzeDUChain()使用修改定义之后的LiveUse和Def集合按照公式求出修改定义之后的LiveOut,从而用它定位块和块之间的DU链。\\\\
BasicBlock.java的修改如下:\\
在函数computeDUDefAndLiveUse(Map<Temp,Set<BasicBlock>>)中,传入的参数是一个空Map的指针,用来记录不同的变量被定值的基本块。我参考了函数computeDefAndLiveUse()的逻辑,对于每一个块,从头到尾遍历语句,遇见定值前引用的变量便保存进LiveUse()中,遇见定值后引用的变量则和最近一次的定值信息一起加入基本块的DU链;遇见定值点,则记录该点在块内定值或者更新最后的定值点,并且将块的指针保存进Map中。在对所有块遍历之后,我们分别求出了不同块的LiveUse信息,块内的DU链;并且求出了程序中所有寄存器的定值位置,从而可以在下一步建立Def集合。\\
在函数computeDEFED(Map<Temp,Set<BasicBlock>>)中,传入的参数和上一个函数相同,通过不同寄存器的定值位置反解求出不同的Def集合;在对所有块遍历之后,它们的Def集合也就求出来了。\\\\
函数analyzeDUChain()在计算得出等效的LiveOut之后调用,按照公式,计算出来的LiveOut包括DU链中的定值元素和它在块外的引用点,综合在第一个函数中求出来的最后一个定值点,便得到块于块之间的DU链。\\
不同的集合和公式说明如下:\\
修改之后的LiveUse(B)为对(s,A)的集合,其中s是块B中某点,s引用变量A的值,且B中在 s前面没有 A 的定值点。\\
修改之后的Def(B)为对(s,A)的集合,其中s是不属于B的某点,s引用变量A的值,但A在B中被重新定值。\\
求值公式如下:
$$LiveIn[B] = LiveUse[B]\bigcup (LiveOut[B]-Def[B])$$
$$LiveOut[B] = \bigcup (LiveIn[b]), b\in S[B]$$
其中S[B]是块B的全部后继。\\
使用公式1,2对于不同的块的LiveIn,LiveOut进行循环赋值,直到得到一个LiveIn和LiveOut的闭包,此时的不同属性均符合要求。\\
FlowGraph.java的修改如下:\\
在函数analyzeDUChain()中实现了上述算法,且在创建类的时候按顺序调用不同BasicBlock中的上述接口和analyzeDUChain(),最终得到一条完整的DU链。\\\\
\textbf{分析样例}\\ 
TestCases/S4/t0.decaf对应的.s文件输出和DU链的分析如下:

\begin{lstlisting}[caption={}]
FUNCTION _Main_New : 
BASIC BLOCK 0 : 
1	_T0 = 4 [ 2 ]	//_T0在第二句中被传参,之后并未出现
2	parm _T0
3	_T1 =  call _Alloc [ 5 6 ]	//_T1在5,6句中被引用,然后并未出现
4	_T2 = VTBL <_Main> [ 5 ]	//_T2在5,6句中被引用,然后并未出现
5	*(_T1 + 0) = _T2
6	END BY RETURN, result = _T1

FUNCTION main : 
BASIC BLOCK 0 : 
7	call _Main.f
8	END BY RETURN, void result

FUNCTION _Main.f : 
BASIC BLOCK 0 : 
9	_T7 = 0 [ 10 ]	//_T7在10句中被引用,然后并未出现
10	_T5 = _T7 [ ]	//_T5在之后并未被引用
11	_T8 = 1 [ 12 ]	//_T8在12句中被引用,然后并未出现
12	_T6 = _T8 [ ]	//_T6在之后并未被引用
13	_T10 = 0 [ 14 ]	//_T10在14句中被引用,然后并未出现
14	_T9 = _T10 [ 21 24 30 ]	//_T9在21,24,30句中被引用,然后并未出现
15	_T11 = 2 [ 16 ]	//_T11在16句中被引用,然后并未出现
16	_T3 = _T11 [ 18 ]	//_T3在18句中被引用,然后在23句被杀死
17	_T12 = 1 [ 18 ]	//_T12在18句中被引用,然后并未出现
18	_T13 = (_T3 + _T12) [ 19 ]	//_T13在19句中被引用,然后并未出现
19	_T4 = _T13 [ 28 ]	//_T4在28句中被引用,然后在29句被杀死(重新赋值)
20	END BY BRANCH, goto 1
BASIC BLOCK 1 : 
21	END BY BEQZ, if _T9 = 
	    0 : goto 7; 1 : goto 2
BASIC BLOCK 2 : 
22	_T14 = 1 [ 23 ]	//_T14在2句中被引用,然后并未出现
23	_T3 = _T14 [ 35 ]//_T3在35句被引用,然后并未出现
24	END BY BEQZ, if _T9 = 
	    0 : goto 4; 1 : goto 3
BASIC BLOCK 3 : 
25	call _Main.f
26	END BY BRANCH, goto 4
BASIC BLOCK 4 : 
27	_T15 = 1 [ 28 ]//_T15在28句被引用,然后并未出现
28	_T16 = (_T4 + _T15) [ 29 ]//_T16在39句被引用,然后并未出现
29	_T4 = _T16 [ 28 32 36 ]//_T4按照4-6的顺序在36句被引用,4-6-4的顺序在28句被引用,4-5的顺序在32句被引用,然后在33句被杀死
30	END BY BEQZ, if _T9 = 
	    0 : goto 6; 1 : goto 5
BASIC BLOCK 5 : 
31	_T17 = 4 [ 32 ]//_T17在32句被引用,然后并未出现
32	_T18 = (_T4 - _T17) [ 33 ]//_T18在33句被引用,然后并未出现
33	_T4 = _T18 [ 28 36 ]//_T4沿着5-6的顺序在36句被引用,沿着5-6-1的顺序在28句被引用,然后在29句被杀死
34	END BY BRANCH, goto 6
BASIC BLOCK 6 : 
35	_T5 = _T3 [ ]//_T5在之后并未出现
36	_T6 = _T4 [ ]//_T6在之后并未出现
37	END BY BRANCH, goto 1
BASIC BLOCK 7 : 
38	END BY RETURN, void result
\end{lstlisting}
和2.2图进行比较,发现:
块1中,前14句是和块之间跳转条件相关的准备;而15,16句的tac对应图中的d1(i:=2);它们DU链的并集减去自己的编号为{18};\\
17,18,19句tac对应图中的d2(j:=i+1),而此处它们DU链的并集为{28};\\
27,28,29句tac对应的是块3中的d4(j:=j+1),此时它们DU链的并集为{28,32,36};\\
31,32,33句tac对应的是块4中的d5(j:=j-4),此时它们DU链的并集为{28,36};\\
35句对应块6中的d6,36句对应块6中的d7,它们的DU链都为空;\\
22,23句的tac对应的是块2中的d3(i := 1),它们的DU链是35,而前面已经提到了,35句对应点d6。\\
综上所诉,可以由这一份t0.du得出:\\
DU[d1]=\{\},DU[d2]=\{d4\},DU[d3]=\{d6\},DU[d4]=\{d4,d5,d7\},DU[d5]=\{d4,d7\},DU[d6]=\{\},DU[d7]=\{\}\\
易得,其和讲义2.4.2节给出的DU链是一致的。
\end{document}